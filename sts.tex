\documentclass[10pt,a4paper,landscape]{article}

% Pakete für kompaktes Layout und deutsche Umlaute
\usepackage[utf8]{inputenc}
\usepackage[ngerman]{babel}
\usepackage[margin=1cm]{geometry}
\usepackage{multicol}
\usepackage{amsmath}
\usepackage{amsfonts}
\usepackage{amssymb}
\usepackage{array}
\usepackage{siunitx}
\usepackage{makecell}
\usepackage{graphicx}
\usepackage{pdfpages}

% Kompakte Darstellung
\setlength{\parindent}{0pt}
\setlength{\parskip}{0.5ex}
\setlength{\columnsep}{1cm}

% Überschriften anpassen
\usepackage{titlesec}
\titleformat{\section}{\normalfont\Large\bfseries}{\thesection}{1em}{}
\titlespacing*{\section}{0pt}{1ex}{0.5ex}
\titleformat{\subsection}{\normalfont\large\bfseries}{\thesubsection}{1em}{}
\titlespacing*{\subsection}{0pt}{0.5ex}{0.25ex}

% Kopf- und Fußzeile entfernen
\pagestyle{empty}

% Redefine section commands to use less space
\makeatletter
\renewcommand{\section}{\@startsection{section}{1}{0mm}%
                                {-1ex plus -.5ex minus -.2ex}%
                                {0.5ex plus .2ex}%x
                                {\normalfont\large\bfseries}}
\renewcommand{\subsection}{\@startsection{subsection}{2}{0mm}%
                                {-1explus -.5ex minus -.2ex}%
                                {0.5ex plus .2ex}%
                                {\normalfont\normalsize\bfseries}}

\renewcommand{\subsubsection}{\@startsection{subsubsection}{3}{0mm}%
                                {-1ex plus -.5ex minus -.2ex}%
                                {0.5ex plus .2ex}%
                                {\normalfont\small\bfseries}}
\makeatother

\setlength{\parindent}{0pt}
\setlength{\parskip}{0pt plus 0.5ex}

\begin{document}

\begin{center}
  \Large\textbf{Stochastik und Statistik}
\end{center}

\begin{multicols*}{3}
  \setlength{\premulticols}{1pt}
  \setlength{\postmulticols}{1pt}
  \setlength{\multicolsep}{1pt}
  \setlength{\columnsep}{2pt}

  \section*{Deskriptive Statistik}
  \subsection*{Begriffe}
  \begin{itemize}
    \item \textbf{Merkmalsträger}: Objekte, über die Informationen gesammelt werden (z.B. Personen, Tiere, Dinge)
    \item \textbf{Merkmal}: Eigenschaft eines Merkmalsträgers (z.B. Alter, Geschlecht, Größe)
    \item \textbf{Grundgesamtheit}: Gesamte Menge aller Merkmalsträger
    \item \textbf{Stichprobe}: Teilmenge der Grundgesamtheit
    \item \textbf{Vollerhebung}: Untersuchung der gesamten Grundgesamtheit
    \item \textbf{Ausprägung}: Mögliche Werte eines Merkmals
  \end{itemize}

  \includegraphics[width=\linewidth]{assets/Markmalstyp.png}

  \subsection*{Relative/Absolute Häufigkeit}
  \subsubsection*{Nicht klassierte Daten}
  \begin{equation*}
    f_i = \frac{h_i}{n}
  \end{equation*}
  \subsubsection*{Klassierte Daten}
  \begin{equation*}
    f_i = \frac{g(x)}{b_i} \quad \text{mit} \quad g(x) = \frac{h_i}{n}
  \end{equation*}
  \begin{itemize}
    \item $f_i$: relative Häufigkeit der Ausprägung $i$ (PMF)
    \item $h_i$: absolute Häufigkeit der Ausprägung $i$ 
    \item $n$: Gesamtanzahl der Beobachtungen
  \end{itemize}

  \subsection*{Summenhäufigkeit}
  \subsubsection*{Nicht klassierte Daten}
  \begin{equation*}
    H(x) = \sum_{j=1}^{x} h_j \quad \text{bzw.} \quad F(x) = \sum_{j=1}^{x} f_j
  \end{equation*}
  \subsubsection*{Klassierte Daten}
  \begin{equation*}
    H(x_k) = \sum_{j=1}^{k} h_j \quad \text{bzw.} \quad F(x) = \int_{-\infty}^{x} f(y) \, dy
  \end{equation*}
  \begin{itemize}
    \item $H(x)$: Summe aller absoluten Häufigkeiten bis Ausprägung $x$
    \item Wenn $h$ absolute Häufigkeit, dann heisst $H$ Summenhäufigkeit
    \item Wenn $f$ relative Häufigkeit, dann heisst $F$ kumulative Verteilungsfunktion (CDF)
  \end{itemize}

  \subsection*{Eigenschaften der PMF und CDF nicht klassierter Daten}
  \begin{itemize}
    \item $0 \leq f_i \leq 1$ und $0 \leq F(x) \leq 1$
    \item $\sum_{i} f_i = 1$ bzw. $F(x_{\max}) = 1$
    \item $F(x)$ ist monoton wachsend $\Rightarrow F(x_1) \leq F(x_2)$ für $x_1 < x_2$
    \item $F(x)$ ist rechtsstetig heisst der Wert an der Stelle $x$ ist gleich dem Grenzwert von rechts
  \end{itemize}

  \subsection*{Eigenschaften der PMF und CDF klassierter Daten}
  \begin{itemize}
    \item $0 \leq f_i \leq 1$ und $0 \leq F(x) \leq 1$
    \item $\int_{-\infty}^{\infty} f(x) \,dx = 1$ und $F'(x) = f(x)$
    \item Für $x \in K_i$ gilt: $\frac{F(x)-F(a_i)}{b_i} = \frac{F(a_{i+1})-F(a_i)}{b_i} = f_i$
  \end{itemize}

  \subsection*{Klassierte Daten}
  \begin{itemize}
    \item Daten werden in Klassen eingeteilt
    \item Klassenbreite: $b_i = a_{i+1} - a_i$ (mit $a_i$ Klassenuntergrenze)
    \item Klassenmitte: $M_i = \frac{a_{i+1} - a_i}{2}$ (mit $l$ untere und $u$ obere Klassenbegrenzung)
    \item Dichte: $d_i = \frac{f_i}{b_i}$
  \end{itemize}

  \subsection*{Kenngrössen}
  \begin{itemize}
    \item \textbf{Arithmetisches Mittel}:
    \begin{equation*}
      \bar{x} = \frac{1}{n} \sum_{i=1}^{n} x_i \quad \text{bzw. wenn klassiert} \quad \bar{x} = \sum_{i=1}^{m} f_i \cdot M_i
    \end{equation*}
    
    \item \textbf{Median}:
    \begin{equation*}
      \tilde{x} = 
      \begin{cases}
        x_{\frac{n+1}{2}}, & n \text{ ungerade} \\
        \frac{x_{\frac{n}{2}} + x_{\frac{n}{2}+1}}{2}, & n \text{ gerade}
      \end{cases}
    \end{equation*}
    \item \textbf{Modalwert}:
    \begin{equation*}
      \hat{x} = x_i \quad \text{mit } h_i = \max(h_1, h_2, \ldots, h_k)
    \end{equation*}
    \item \textbf{Varianz}:
    \begin{equation*}
      \tilde{s}^2 = \frac{1}{n} \sum_{i=1}^{n} {(x_i - \bar{x})}^2 \quad \text{(bei klassierten Daten $x_i = M_i$)}
    \end{equation*}
    \begin{equation*}
      \tilde{s}^2 = \overline{x^2} - \bar{x}^2 \quad \text{mit} \quad \overline{x^2} = \frac{1}{n} \sum_{i=1}^{n} x_i^2 
    \end{equation*}
    \begin{equation*}
      s^2 = \frac{1}{n-1} \sum_{i=1}^{n} {(x_i - \bar{x})}^2 \quad  \text{(bzw.)} \quad s^2 = \frac{n}{n-1} \tilde{s}^2
    \end{equation*}
    \item \textbf{Standardabweichung}:
    \begin{equation*}
      s = \sqrt{s^2} \quad \text{bzw.} \quad \tilde{s} = \sqrt{\tilde{s}^2}
    \end{equation*}
    \item \textbf{Spannweite}:
    \begin{equation*}
      R = x_{\max} - x_{\min}
    \end{equation*}
    \item \textbf{Quantile}:
    \begin{equation*}
      Q_p = 
      \begin{cases}
        x_{k}, & \text{für } n \cdot p \text{ ganzzahlig} \\
        x_{\lceil n \cdot p \rceil}, & \text{sonst}
      \end{cases}
    \end{equation*}
  \end{itemize}

  \includegraphics[width=\linewidth]{assets/Boxplot.png}

  \columnbreak

  \subsection*{Kovarianz und Korrelation}
  \begin{itemize}
    \item \textbf{Kovarianz}:
    \begin{equation*}
      \tilde{s}_{xy} = \text{Cov}(X,Y) = \frac{1}{n} \sum_{i=1}^{n} (x_i - \bar{x})(y_i - \bar{y}) = \overline{xy} - \bar{x} \cdot \bar{y}
    \end{equation*}
    \item \textbf{Korrelationskoeffizient nach Pearson}:
    \begin{equation*}
      r = \frac{\text{Cov}(X,Y)}{s_X s_Y} = \frac{\tilde{s}_{xy}}{\sqrt{\overline{x^2} - \bar{x}^2} \cdot \sqrt{\overline{y^2} - \bar{y}^2}}
    \end{equation*}
    \item $r$ liegt im Intervall $[-1, 1]$
    \item $r > 0$: positive Korrelation, $r < 0$: negative Korrelation, $r = 0$: keine Korrelation
    \item $\tilde{s}_{xy}$ und $r$ beschreiben die lineare Abhängigkeit zwischen zwei Merkmalen d.h. wie stark die Merkmale von der Geraden $y = mx + b$ abweichen
    \item Ist nicht robust gegenüber Ausreissern
  \end{itemize}

  \subsection*{Rangkorrelation nach Spearman}
  \begin{itemize}
    \item Rang:
    {\small
    \begin{equation*}
      rg(x_i) = 
      \begin{cases}
        k, & \text{wenn } x_i \text{ der } \\ & k\text{-te kleinste Wert ist} \\
        \frac{1}{m} \sum_{j=1}^{m} rg(x_{i_j}), & \text{bei } m \text{ gleichen Werten}
      \end{cases}
    \end{equation*}
  }
    \item Korrelationskoeffizient nach Spearman:
    \begin{equation*}
      r_s = 1 - \frac{6 \sum_{i=1}^{n} d_i^2}{n(n^2 - 1)} \quad \text{mit } d_i = rg(x_i) - rg(y_i)
    \end{equation*}
    \item $r_s$ beschreibt die monotone Abhängigkeit zwischen zwei Merkmalen
    \item Ist robust gegenüber Ausreissern
  \end{itemize}

  \section*{Kombinatorik}
  \subsection*{Grundlagen}
  \begin{itemize}
    \item \textbf{Multiplikationsregel}: Wenn ein Vorgang in $m$ Arten und ein zweiter Vorgang in $n$ Arten durchgeführt werden kann, dann können beide Vorgänge in $m \cdot n$ Arten durchgeführt werden.
    \item \textbf{Additionsregel}: Wenn ein Vorgang in $m$ Arten und ein zweiter Vorgang in $n$ Arten durchgeführt werden kann, und beide Vorgänge sich gegenseitig ausschließen, dann können beide Vorgänge in $m + n$ Arten durchgeführt werden.
  \end{itemize}
  \subsection*{Binomialkoeffizient}
  \begin{equation*}
    \binom{n}{k} = \frac{n!}{k!(n-k)!}
  \end{equation*}
  \begin{itemize}
    \item Anzahl der Möglichkeiten, aus $n$ Elementen $k$ Elemente ohne Zurücklegen und ohne Beachtung der Reihenfolge auszuwählen
    \item $n$: Gesamtanzahl der Elemente
    \item $k$: Anzahl der auszuwählenden Elemente
    \item $\binom{n}{0} = 1$, $\binom{n}{n} = 1$, $\binom{n}{1} = n$
    \item Symmetrie: $\binom{n}{k} = \binom{n}{n-k}$
    \item Rekursion: $\binom{n}{k} = \binom{n-1}{k-1} + \binom{n-1}{k}$
    \item Binomischer Lehrsatz: $(x + y)^n = \sum_{k=0}^{n} \binom{n}{k} x^{n-k} y^k$
    \item Summe der Binomialkoeffizienten: $\sum_{k=0}^{n} \binom{n}{k} = 2^n$
  \end{itemize}

\begin{center}
\textbf{Kombinatorische Auswahlmöglichkeiten}
\end{center}

\setlength{\tabcolsep}{9pt}
\renewcommand{\arraystretch}{1.3}
\begin{tabular}{|>{\centering\arraybackslash}m{0.28\linewidth}
                |>{\centering\arraybackslash}m{0.28\linewidth}
                |>{\centering\arraybackslash}m{0.28\linewidth}|}
\hline
 & \textbf{Mit Zurücklegen} & \textbf{Ohne Zurücklegen} \\ \hline
  \textbf{Mit Reihenfolge} & $n^k$ & $\frac{n!}{(n-k)!}$ \\
    & \textit{Zahlenschloss} & \textit{Rennen} \\ \hline
    
  \textbf{Ohne Reihenfolge} & $\binom{n+k-1}{k}$ & $\binom{n}{k}$ \\
    & \textit{Warenwahl} & \textit{Gruppenwahl} \\ \hline
\end{tabular}

\end{multicols*}

\includepdf[pages=-, fitpaper=true]{assets/STS_TabellenVerteilungen.pdf}
\end{document}
